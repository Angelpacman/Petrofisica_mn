\documentclass[12pt]{book}
%%%%%%%%%%%%%%%%%%%%%%%%%%
%%
%%preambulo

\usepackage[T1]{fontenc}
\usepackage[utf8]{inputenc}
\usepackage[spanish]{babel}
%%

\begin{document}    
\title{Evaluación Petrofísica avanzada y Aprendizaje Máquina como herramienta de visualización}
\author{José Angel Resendiz Avilés}
\date{3 de Febrero 2020}
\maketitle

\chapter{Parametros litológicos M y N}

\paragraph{Intoducción a las secuencias sedimentarias}

\textnormal{Los registros geofisicos se pueden describir de una manera sencilla si pensamos en un muestro de de datos correspondientes a propiedades fisicas que se comprende en una dimension a traves del subsuelo}

\end{document}
